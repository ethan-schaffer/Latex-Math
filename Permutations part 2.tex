% !TEX TS-program = pdflatex
% !TEX encoding = UTF-8 Unicode

% This is a simple template for a LaTeX document using the "article" class.
% See "book", "report", "letter" for other types of document.

\documentclass[11pt]{article} % use larger type; default would be 10pt

\usepackage[utf8]{inputenc} % set input encoding (not needed with XeLaTeX)

%%% Examples of Article customizations
% These packages are optional, depending whether you want the features they provide.
% See the LaTeX Companion or other references for full information.

%%% PAGE DIMENSIONS
\usepackage{geometry} % to change the page dimensions
\geometry{a4paper} % or letterpaper (US) or a5paper or....
% \geometry{margin=2in} % for example, change the margins to 2 inches all round
% \geometry{landscape} % set up the page for landscape
%   read geometry.pdf for detailed page layout information

\usepackage{graphicx} % support the \includegraphics command and options

% \usepackage[parfill]{parskip} % Activate to begin paragraphs with an empty line rather than an indent

%%% PACKAGES
\usepackage{tikz}
\usepackage{booktabs} % for much better looking tables
\usepackage{color} % Added by me
\usepackage{array} % for better arrays (eg matrices) in maths
\usepackage{paralist} % very flexible & customisable lists (eg. enumerate/itemize, etc.)
\usepackage{verbatim} % adds environment for commenting out blocks of text & for better verbatim
\usepackage{subfig} % make it possible to include more than one captioned figure/table in a single float
% These packages are all incorporated in the memoir class to one degree or another...

%%% HEADERS & FOOTERS
\usepackage{fancyhdr} % This should be set AFTER setting up the page geometry
\pagestyle{fancy} % options: empty , plain , fancy
\renewcommand{\headrulewidth}{0pt} % customise the layout...
\lhead{}\chead{}\rhead{}
\lfoot{}\cfoot{\thepage}\rfoot{}

%%% SECTION TITLE APPEARANCE
\usepackage{sectsty}
\allsectionsfont{\sffamily\mdseries\upshape} % (See the fntguide.pdf for font help)
% (This matches ConTeXt defaults)

%%% ToC (table of contents) APPEARANCE
\usepackage[nottoc,notlof,notlot]{tocbibind} % Put the bibliography in the ToC
\usepackage[titles,subfigure]{tocloft} % Alter the style of the Table of Contents
\renewcommand{\cftsecfont}{\rmfamily\mdseries\upshape}
\renewcommand{\cftsecpagefont}{\rmfamily\mdseries\upshape} % No bold!

%%% END Article customizations

%%% The "real" document content comes below... %%%

\title{Permutations}
\author{Ethan Schaffer}
\date{Due 10/28/16}
\newcommand\tab[1][1cm]{\hspace*{#1}}

\begin{document}
\maketitle
\section* {Problem}
\textit{Find and prove a formula for computing $(n)_k.$}

\section*{Solution}
I know that  $(n)_k$ means ``the number of permutations of n objects k at a time''. In other words, I am picking from a group of size n, but only until I have k of them. To find a formula, I compared n! with what I was looking for using an example of $(4)_2$. I knew that 4! was 4*3*2*1 while I was really looking for 4*3. I tried again with the example of $(6)_3$, and found that 6! was too large, and I was really looking for 6*5*4. I was always ending up with extra factors. Then, I realized that if I take $\frac{n!}{(n-k)!}$, I would effectively eliminate the extra factors of n!. I tried it for my prior examples, as well as some others, and found that \begin{equation}\frac{n!}{(n-k)!}\end{equation} was exactly what I was looking for.
 
\section* {Analysis}
This problem is pretty basic, which was a good break of pace from yesterday's assignment.

\section* {Status}
\tab I feel like I fully understand this problem.

\section* {MVP}
\tab I have no MVP for this problem
\end{document}
