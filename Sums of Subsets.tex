% !TEX TS-program = pdflatex
% !TEX encoding = UTF-8 Unicode

% This is a simple template for a LaTeX document using the "article" class.
% See "book", "report", "letter" for other types of document.

\documentclass[11pt]{article} % use larger type; default would be 10pt

\usepackage[utf8]{inputenc} % set input encoding (not needed with XeLaTeX)

%%% Examples of Article customizations
% These packages are optional, depending whether you want the features they provide.
% See the LaTeX Companion or other references for full information.

%%% PAGE DIMENSIONS
\usepackage{geometry} % to change the page dimensions
\geometry{a4paper} % or letterpaper (US) or a5paper or....
% \geometry{margin=2in} % for example, change the margins to 2 inches all round
% \geometry{landscape} % set up the page for landscape
%   read geometry.pdf for detailed page layout information

\usepackage{graphicx} % support the \includegraphics command and options

% \usepackage[parfill]{parskip} % Activate to begin paragraphs with an empty line rather than an indent

%%% PACKAGES
\usepackage{tikz}
\usepackage{booktabs} % for much better looking tables
\usepackage{color} % Added by me
\usepackage{array} % for better arrays (eg matrices) in maths
\usepackage{paralist} % very flexible & customisable lists (eg. enumerate/itemize, etc.)
\usepackage{verbatim} % adds environment for commenting out blocks of text & for better verbatim
\usepackage{subfig} % make it possible to include more than one captioned figure/table in a single float
% These packages are all incorporated in the memoir class to one degree or another...

%%% HEADERS & FOOTERS
\usepackage{fancyhdr} % This should be set AFTER setting up the page geometry
\pagestyle{fancy} % options: empty , plain , fancy
\renewcommand{\headrulewidth}{0pt} % customise the layout...
\lhead{}\chead{}\rhead{}
\lfoot{}\cfoot{\thepage}\rfoot{}

%%% SECTION TITLE APPEARANCE
\usepackage{sectsty}
\allsectionsfont{\sffamily\mdseries\upshape} % (See the fntguide.pdf for font help)
% (This matches ConTeXt defaults)

%%% ToC (table of contents) APPEARANCE
\usepackage[nottoc,notlof,notlot]{tocbibind} % Put the bibliography in the ToC
\usepackage[titles,subfigure]{tocloft} % Alter the style of the Table of Contents
\renewcommand{\cftsecfont}{\rmfamily\mdseries\upshape}
\renewcommand{\cftsecpagefont}{\rmfamily\mdseries\upshape} % No bold!

%%% END Article customizations

%%% The "real" document content comes below... %%%

\title{Sums of Subsets}
\author{Ethan Schaffer}
\date{Due 11/8/16}
\newcommand\tab[1][1cm]{\hspace*{#1}}

\begin{document}
\maketitle
\section* {Problem}
\textit{Explain why:}
\begin{equation}
2^n = ({n\atop 0}) + ({n\atop 1}) + ({n\atop 2}) + ... + ({n\atop n-1}) + ({n\atop n})
\end{equation}

\section*{Solution}
\tab To solve this problem, I will prove that both the left and right side of the equation are solutions to the same problem. Upon doing so, I will know for certain that the left and right sides of the equation are equal. 
\\ \tab We have already proven that the number of subsets that can exist for a set of size n is $2^n$, so if the right side of the equation is also an expression of how many subsets can exists for a set of size n, then the two sides of the equation must be equal. 
\\ \tab We also know that $({n\atop 0})$ is the number of subsets for which 0 objects are picked from a set of size n, and that $({n\atop 1})$ is the same, but when 1 item is chosen. So, for a set of size 1, we see that $({n\atop 0})$+$({n\atop 1})$ is 2, just like $2^1$ is. This smaller example where n=1 is obvious, but the same logic goes when n is a larger value. After all, when we take the sum of every possible number of subset we can make of size k, we would get the total number of possible subsets for our set of size n. 
\\ \tab Because both the left and right side of the formula calculate the same thing, they MUST be equal. 

\section* {Analysis}
\tab This problem was interesting, but it was pretty anti-climactic. It seemed to be missing an epiphany of any sort\footnote{At least that I found}.

\section* {Status}
\tab I feel like I fully understand this problem.

\section* {MVP}
\tab I have no MVP for this problem.
\end{document}
