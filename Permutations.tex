% !TEX TS-program = pdflatex
% !TEX encoding = UTF-8 Unicode

% This is a simple template for a LaTeX document using the "article" class.
% See "book", "report", "letter" for other types of document.

\documentclass[11pt]{article} % use larger type; default would be 10pt

\usepackage[utf8]{inputenc} % set input encoding (not needed with XeLaTeX)

%%% Examples of Article customizations
% These packages are optional, depending whether you want the features they provide.
% See the LaTeX Companion or other references for full information.

%%% PAGE DIMENSIONS
\usepackage{geometry} % to change the page dimensions
\geometry{a4paper} % or letterpaper (US) or a5paper or....
% \geometry{margin=2in} % for example, change the margins to 2 inches all round
% \geometry{landscape} % set up the page for landscape
%   read geometry.pdf for detailed page layout information

\usepackage{graphicx} % support the \includegraphics command and options

% \usepackage[parfill]{parskip} % Activate to begin paragraphs with an empty line rather than an indent

%%% PACKAGES
\usepackage{tikz}
\usepackage{booktabs} % for much better looking tables
\usepackage{color} % Added by me
\usepackage{array} % for better arrays (eg matrices) in maths
\usepackage{paralist} % very flexible & customisable lists (eg. enumerate/itemize, etc.)
\usepackage{verbatim} % adds environment for commenting out blocks of text & for better verbatim
\usepackage{subfig} % make it possible to include more than one captioned figure/table in a single float
% These packages are all incorporated in the memoir class to one degree or another...

%%% HEADERS & FOOTERS
\usepackage{fancyhdr} % This should be set AFTER setting up the page geometry
\pagestyle{fancy} % options: empty , plain , fancy
\renewcommand{\headrulewidth}{0pt} % customise the layout...
\lhead{}\chead{}\rhead{}
\lfoot{}\cfoot{\thepage}\rfoot{}

%%% SECTION TITLE APPEARANCE
\usepackage{sectsty}
\allsectionsfont{\sffamily\mdseries\upshape} % (See the fntguide.pdf for font help)
% (This matches ConTeXt defaults)

%%% ToC (table of contents) APPEARANCE
\usepackage[nottoc,notlof,notlot]{tocbibind} % Put the bibliography in the ToC
\usepackage[titles,subfigure]{tocloft} % Alter the style of the Table of Contents
\renewcommand{\cftsecfont}{\rmfamily\mdseries\upshape}
\renewcommand{\cftsecpagefont}{\rmfamily\mdseries\upshape} % No bold!

%%% END Article customizations

%%% The "real" document content comes below... %%%

\title{Counting}
\author{Ethan Schaffer}
\date{Due 10/27/16}
\newcommand\tab[1][1cm]{\hspace*{#1}}

\begin{document}
\section* {2.1.1}
\textit{In Minnesota, the license plates have three letters followed by three numbers or three numbers followed by three letters. How many possible license plates are there?}
The number of possible licence plates is:
\begin{equation} 2*(26^3*10^3) \end{equation}
I reached this by knowing that there are $26^3$ possible 3 letter sets, and $10^3$ possible 3 number sets. 
However, because we can either have letters of numbers first, we have to multiply their product by 2. 

\textbf{In the following exercises, we will assume that the license plates are three
letters followed by three numbers.}

\section*{2.1.2.}
\textit{How many license plates have all distinct numbers and letters?}
The number of possible licence plates with all distinct numbers and letters is:
\begin{equation} (26*25*24)(10*9*8) \end{equation}
This can also be expressed as:
\begin{equation} \frac{(26!)}{(26-3)!}*\frac{(10!)}{(10-3)!} \end{equation}
I find this interesting because there are 3 letters and 3 numbers. 

\section*{2.1.3}
\textit{How many license plates have a “double letter”, that is, two adjacent equal letters? A “double number”? (Note: triple letters are also double letters and triple numbers are also double numbers.)}
The number of possible licence plates with no double letters is:
\begin{equation} (26^3  - 26*25*25)10^3 \end{equation}

The number of possible licence plates with no double numbers is:
\begin{equation} (10^3  - 10*9*9)26^3 \end{equation}

\section*{2.1.4}
\textit{How many license plates have a double letter and a double number? A double letter or a double number?}
The number of possible licence plates with a double letter \textbf{AND} a double number is:
\begin{equation} (10^3-(10*9*9)) * (26^3-(26*25*25)) \end{equation}
The number of possible licence plates with a double letter \textbf{OR} a double number is:

\begin{equation} (26^3  - 26*25*25)10^3 + \end{equation} \begin{equation}(10^3  - 10*9*9)26^3  - \end{equation} \begin{equation} (10^3-(10*9*9)) * (26^3-(26*25*25))\end{equation}

\newpage
\textbf{Suppose there are 15 different apples and 10 different pears. How many ways
are there for Jack to pick an apple or a pear and then for Jill to pick an apple
or a pear? The “or” connectors tell us that an addition is involved. The “and
then” connector tells us a multiplication is involved. Jack has 25 different fruit
to choose from (15 + 10). Jill then has 24 different fruit to choose from (since
Jack has already taken one). So the number of ways is 25*24.
In Exercises 2.1.5 to 2.1.7, pay particular attention to the conjunctions “and”
and “or.”}

\section*{2.1.5} \textit{How many ways are there for Jack to pick an apple and a pear
and then for Jill to pick an apple and a pear?}
The number of ways for Jack to pick an apple and a pear and then for Jill to pick an apple and a pear is:
\begin{equation} (15*10)*(14*9) \end{equation}
Jack can pick any 1 of 15 apples, then any 1 of 10 pears. Then, Jill can pick any 1 of 14 apples, then any 1 of 9 pears. 


\section*{2.1.6} \textit{How many ways are there for Jack to pick an apple or a pear
and then for Jill to pick an apple and a pear?}
Jack can pick any 1 of 25 fruits. Then, Jill makes different picks based on Jack's choice. 
She can either pick any 1 of 15 apples, then any 1 of 9 pears or any 1 of 14 apples, then any 1 of 10 pears. 
I am not sure how to express that mathematically, but my best guess is:
\begin{equation} (25)*(15*9+14*10-(15+10)) \end{equation}

\section*{2.1.7} \textit{How many ways are there for Jack to pick an apple and a pear
and then for Jill to pick an apple or a pear?}
\begin{equation} (15*10)*(23) \end{equation}
Jack can pick any 1 of 15 apples, then any 1 of 10 pears. Then, Jill can pick any 1 of 23 fruits. 

\newpage
\maketitle
\section* {Problem}
\textit{How many subsets of a set of size n?}

\section*{Solution}
\tab I knew that based on the \textbf{One to One Coorespondence Principle}, if I could find a set of rules that have one to one coorespondence, I can find a formula that relates to that set of rules. My new set of rules was to make a space for each possible element of the subset. Then, I would either put in a 1 or 0, indicating whether or not the cooresponding value was in my subset. The values I choose to put there do not matter as much as how many options I have. Applying the One to One Coorespondence Principle, I can write this as a set of n rules, where n in the size of my set. Each rule would simply be ''Is a there?'', where a is a value of the original set. Therefore, for a set of size 3, I would have to answer 3 yes or no questions. Therefore, I would have 2*2*2, or $2^3$, possible solutions. Based on this, I concluded that for a set of size n, I would have $2^n$ possible solutions. 
 
\section* {Analysis}
This problem was interesting, and significantly easier that 4.2.1 and 4.2.6. However, it was still an interesting proof to work through.

\section* {Status}
\tab I feel like I fully understand this problem.

\section* {MVP}
\tab My MVP is Matthew. He helped me more with problems 2.1.1-2.1.7 than for problem 2.1.8, but his help was still awesome. 

\end{document}
